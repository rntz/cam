\documentclass{article}
\usepackage[a4paper,margin=0.8in]{geometry}

\usepackage{amssymb,amsmath,amsthm}
\usepackage{latexsym}
\usepackage{url}
\usepackage{hyperref}

\usepackage{multirow}

%% uses the AMS Euler math font.
\usepackage{euler}

%% Put space between paragraphs rather than indenting.
%% \usepackage{parskip}

%% \setlength{\parindent}{0pt}
%% \setlength{\parskip}{0.6em}

%% %% for coloneqq
%% \usepackage{mathtools}

%% %% for \underaccent
%% \usepackage{accents}

%% % for semantic brackets
%% \usepackage{stmaryrd}

%% % for inference rules
%% \usepackage{proof}

%% %% for mathpar environment
%% \usepackage{mathpartir}

%% % for \scalebox, \rotatebox
%% \usepackage{graphicx}

%% for colors
%% must be loaded before tikz in order for options to take effect :/
\usepackage[dvipsnames]{xcolor}

%% \usepackage{tikz}
%% \usepackage{tikz-cd}


\newcommand{\fa}[1]{\forall\,{#1}.\ }
\newcommand{\te}[1]{\exists\,{#1}.\ }

\newcommand{\N}{\mathbb{N}}
\newcommand{\ms}[1]{\ensuremath{\textsf{#1}}}
\newcommand{\setof}[2]{\{{#1}~|~{#2}\}}

\newcommand{\todo}[1]{{\color{red}{#1}}}

\newtheorem{theorem}{Theorem}
\newtheorem{corollary}{Corollary}
\newtheorem{nontheorem}{Non-theorem}
\newenvironment{refutation}
  {\vspace{0.5em}\noindent\textit{Refutation.}}
  {\hfill$\square$\vspace{0.5em}}


\begin{document}

\title{Consistency as Monotonicity}
\author{Michael Arntzenius}
\date{}
\maketitle


\section{Abstract}

I sketch a very general model of concurrent computation: a set of functions
iteratively applied in a nondeterministic order. A few simple restrictions, most
importantly monotonicity, suffice to ensure eventual consistency or determinism.


\section{Less abstract}
\label{sec:less-abstract}

A program $P$ is a set of monotone maps on a poset $S$ of program states
with a least element (initial state) $\bot$. A state is ``fixed'' if it is a
fixed point of every function in $P$. A schedule is an infinite sequence of
elements of $P$. A schedule is fair if it visits every $f \in P$ infinitely
often. A schedule terminates if it has a prefix that reaches a fixed state. I
show that a program either:

\begin{enumerate}
\item Diverges: no schedules terminate.
\item Converges: all fair schedules terminate, and all terminating schedules
  reach the same fixed state.
\end{enumerate}


\section{Example: Monotone dataflow}

Here I show how a slightly more concrete model of computation --- monotone
dataflow --- can be represented in this framework.

A monotone dataflow program is a finite, directed, possibly cyclic graph $G =
(V,E)$. Nodes $f \in V$ are monotone functions of 0 or more arguments. $f$'s
in-edges correspond to its arguments. Each edge $e \in E$ has a type, which is a
poset with a least element. This type must agree with the result type of its
source and with the argument type of its destination. A program state is a
labelling of edges with values of the appropriate types. The initial state
labels each edge with the least element of its type.

To ``step'' a subset $S$ of nodes, for each $f \in S$ compute $f$ applied to
the current labels of its in-edges. Then simultaneously update each out-edge of each $f \in S$ with the values thus computed.

\todo{Show how to interpret this in terms of Section \ref{sec:less-abstract}.}


\section{Things this reminds me of}

\begin{enumerate}
\item Consistency As Logical Monotonicity; hence the name.
\item Sussman's propagators.
\item LVars, vaguely.
\item Datafun, of course. Datafun restricts $S$ to satisfy an ascending chain
  condition and $P$ to be a single function. Consequently it gains termination.
\item This feels vaguely like a single-player game semantics. I don't know
  whether that even makes sense.
\end{enumerate}


\section{Definitions}

Fix $S$, a partially ordered set with a least element $\bot$.

$\vec{s} : P$ means $\vec{s}$ is a finite sequence such that $\fa{i} s_i \in P$.

%% \newcommand{\seq}[1]{{\vec{#1}_*}}
\newcommand{\seq}[1]{{\bullet\vec{#1}}}

Given a sequence $\vec{f} = f_0, ..., f_n$, let $\seq{f} = f_n \circ f_{n-1}
\circ ... \circ f_1 \circ f_0$.

Consider sets $P,Q$ of monotone maps $S \to S$. Preorder such sets by:
\[
\begin{array}{ccccc}
P \le Q
&\text{iff}& \fa{f \in P} \te{\vec{s} : Q} f \le \seq{s}
&\text{or equivalently}& \fa{\vec{s} : P} \te{\vec{t} : Q}
{\seq{s} \le \seq{t}}\\
P \simeq Q &\text{iff}& P \le Q \wedge Q \le P
\end{array}
\]

\todo{Could I instead (or additionally) consider sets closed under composition
  and identity? Hm...}

\newcommand{\dec}[1]{\ensuremath{{#1}_{\text{dec}}}}
\newcommand{\inc}[1]{\ensuremath{{#1}_{\text{inc}}}}
\newcommand{\fixed}[1]{\ensuremath{{#1}_{\text{fix}}}}
\newcommand{\reach}[1]{\ensuremath{{#1}_{\text{reach}}}}
%% I'm not sure what the * in newcommand* does.
\newcommand*{\defeq}{\stackrel{\text{def}}{=}}

Given a program $P$, we define the following subsets of $S$:
\[
\begin{array}{ccccc}
  x \in \reach{P}
  &\multirow{4}{*}{\text{iff}}& \te{\vec{s} : P} x = \seq{s}\;\bot
  &\multirow{4}{*}{which is read}&\text{``$x$ is reachable''}\\
  x \in \dec{P} && \fa{f \in P}{f\;x \le x} && \text{``$x$ is decreasing''}\\
  x \in \inc{P} && \fa{f \in P}{f\;x \ge x} && \text{``$x$ is increasing''}\\
  x \in \fixed{P} && \fa{f \in P}{f\;x = x} && \text{``$x$ is fixed''}
\end{array}
\]

\todo{Do we need to consider increasing points?}

Note that $\fixed{P} = \dec{P} \cap \inc{P}$.

\newcommand{\eval}{\Downarrow}
\newcommand{\diverge}{\not\eval}

Write $P \eval x$ (``$P$ evaluates to $x$'') for $x \in \reach{P} \cap \dec{P}$.

Write $P \diverge$ (``$P$ diverges'') for $\reach{P} \cap \dec{P} = \emptyset$.


\section{Proofs}

Initially, I conjectured the following:

\begin{nontheorem}
  If $P \simeq Q$ then $\fixed{P} = \fixed{Q}$.
\end{nontheorem}

However, this is false:

\begin{refutation}
  Let $S = \N$ ordered $0 < 1 < 2 < ...$. Let $P = \{id\}$. Let $Q = \{id,
  pred\}$, where $pred\; 0 = 0$ and $pred\; (x+1) = x$. Then $P \simeq Q$; the
  only interesting case is $pred \le id$. Yet $\fixed{P} = \N$, but $\fixed{Q} =
  \{0\}$.
\end{refutation}

The idea of decreasing points (aka ``prefix points'') as a generalization of
fixed points was introduced to me by Paul Blain Levy. Decreasing points solve
this problem:

\begin{theorem}
  If $P \le Q$ then $\dec{Q} \subseteq \dec{P}$.
  \label{thm:dec-antitone}
\end{theorem}

\begin{proof}
  Assume $x \in \dec{Q}$ and $f \in P$. We wish to show $f\;x \le x$. By $P
  \le Q$ we have $f \le \seq{s}$ for some $\vec{s} : Q$. So $f\;x \le
  \seq{s}\;x$. Since $x \in \dec{Q}$, $\seq{s}\;x \le x$. So $f\; x \le x$.
\end{proof}

\todo{Maybe the ordering on programs is the opposite of what it should be?}

\begin{corollary}
  If $P \simeq Q$, then $\dec{P} = \dec{Q}$.
  \label{cor:simeq->dec}
\end{corollary}

This is interesting, but isn't it fixed points we really care about? As it turns
out, there is a powerful connection between decreasing points and fixed points:

\begin{theorem}
  A least decreasing point is a least fixed point.
\end{theorem}

\begin{proof}
  Since all fixed points are decreasing points, a least decreasing point is less
  than all fixed points. So it suffices to show that a least decreasing point is
  a fixed point.

  Let $x$ be a least decreasing point. Consider any $f \in P$. We wish to show
  $f\;x = x$. We have $f\; x \le x$. We will show that $f\;x$ is also a
  decreasing point. From this, since $x$ is the least decreasing point, we have
  $x \le f\;x$ and therefore $f\;x = x$.

  Consider any $g \in P$. We wish to show that $g\;(f\;x) \le f\; x$. We have
  $f\;x \le x$. Monotonically, $g\;(f\;x) \le g\;x$. Since $x$ is a decreasing
  point, $g\;x \le x$. Transitively, $g\;(f\;x) \le x$ as desired.
\end{proof}

Now we consider the interaction of reachability and decreasing-ness:

\begin{theorem}
  If $x \in \reach{P}$ and $y \in \dec{P}$ then $x \le y$.
  \label{thm:reachable<=dec}
\end{theorem}

\begin{proof}
  Since $x$ is reachable, $x = \seq{s}\; \bot$. Induct on $\vec{s}$. Base case:
  $\bot \le y$. Now inductively consider some $\vec{t}$ such that $\seq{t}\;
  \bot \le y$ and fix $f \in P$. We wish to show that $f\; (\seq{t}\; \bot) \le
  y$. By our IH and monotonicity of $f$, $f\;(\seq{t}\; \bot) \le f\; y$. By $y
  \in \dec{P}$, $f\;y \le y$. Transitively $f\; (\seq{t}\; \bot) \le y$ as
  desired.
\end{proof}

\begin{corollary}
  If $P \eval x$ then $x$ is the least fixed point of $P$.
\end{corollary}

\begin{corollary}[Consistency]
  If $P \eval x$ and $P \eval y$ then $x = y$.
  \label{cor:consistency}
\end{corollary}

\todo{Couldn't I prove a perhaps more traditional ``confluence''-style theorem
  as well?}

\begin{theorem}[Equivalence]
  If $P \simeq Q$ and $P \eval x$ then $Q \eval x$.
  \label{thm:equivalence}
\end{theorem}

\begin{proof}
  Since $x \in \dec{P}$ and $Q \le P$, by Theorem \ref{thm:dec-antitone} we know
  $x \in \dec{Q}$. So it suffices to show that $x \in \reach{Q}$.

  $x = \seq{s}\;\bot$ for some $\vec{s} : P$. So by $P \le Q$ there is some
  $\vec{t} : Q$ such that $\seq{s} \le \seq{t}$ and so $x = \seq{s}\; \bot \le
  \seq{t}\;\bot$. Since $\seq{t}\;\bot \in \reach{Q}$ and $x \in \dec{Q}$, by
  Theorem \ref{thm:reachable<=dec} we have $\seq{t}\;\bot \le x$. By
  antisymmetry $x = \seq{t}\;\bot$ and so $x \in \reach{Q}$.
\end{proof}


%% \begin{theorem}
%%   If $x \in \reach{P}$ and $F : \N \to P$ is fair, then $F$ has a prefix
%%   $\vec{t}$ such that $x \le \vec{t}$.
%% \end{theorem}
%% \begin{proof}
%%   There is a $\vec{s}$ such that $x = \seq{s}\;\bot$. Then, because $F$ is fair,
%%   there are some $j_1 < ... < j_n$ such that $\fa{i}{s_i = F\; j_i}$. Let
%%   $\vec{t} = F\;0, ..., F\;j_n$ be the prefix of $F$ ending at $j_n$.

%%   We induct on $\vec{s}$ to show $\seq{s}\;\bot \le \seq{t}\;\bot$. Base case:
%%   $\bot \le \bot$. Inductively consider some $\vec{s}$ of length $k$ such that
%%   $\seq{s}\;\bot \le F\;j_k\;(...\; (F\; 1\; (F\; 0\; \bot))...)$. Fix $f \in P$. We wish to show that $f\; (\seq{s}\; \bot) \le $

%%   \todo{I'm not even convinced this is true anymore! Try to find a
%%     counterexample!}
%% \end{proof}

For proving that fair schedulings can find reachable fixed points if they exist,
I thought to use the following:

\begin{nontheorem}
  If $x \in \reach{P}$ and $F : \N \to P$ is fair, then $F$ has a prefix
  $\vec{t}$ such that $x \le \vec{t}$.
\end{nontheorem}

But this turns out to be false:

\begin{refutation}
  Let $S$ be finite binary strings, ordered $x \le y$ iff $x$ is a prefix of
  $y$. Let $P = \{\text{append }0, \text{append }1\}$. Then schedules correspond
  precisely with infinite binary strings. But there is no infinite binary string
  such that all finite binary strings are prefixes of it.

  (In this case fair schedules happen to correspond with infinite binary strings
  which have infinitely many 0s and infinitely many 1s. But fairness turns out
  to be a red herring here.)
\end{refutation}

%% \begin{corollary}[Fairness works]
%%   If $P \eval x$ and $S : \N \to P$ is fair, then $S$ has a prefix $\vec{s} =
%%   S\;0, S\;1, ..., S\;n$ such that $x = \seq{s}\;\bot$.
%% \end{corollary}

%% \begin{proof}
%%   It suffices to find a prefix $\vec{s}$ of $S$ such that $x \le \seq{s}\;\bot$;
%%   for then by Theorem \ref{thm:reachable<=dec} we have $\seq{s}\;\bot \le x$
%%   and so $x = \seq{s}\;\bot$.

%%   \todo{TODO}
%% \end{proof}

\end{document}
