\documentclass{article}
\usepackage[a4paper,margin=0.8in]{geometry}

\usepackage{amssymb,amsmath,amsthm}
\usepackage{latexsym}
\usepackage{url}
\usepackage{hyperref}

%% uses the AMS Euler math font.
\usepackage{euler}

%% Put space between paragraphs rather than indenting.
%% \usepackage{parskip}

%% \setlength{\parindent}{0pt}
%% \setlength{\parskip}{0.6em}

%% %% for coloneqq
%% \usepackage{mathtools}

%% %% for \underaccent
%% \usepackage{accents}

%% % for semantic brackets
%% \usepackage{stmaryrd}

%% % for inference rules
%% \usepackage{proof}

%% %% for mathpar environment
%% \usepackage{mathpartir}

%% % for \scalebox, \rotatebox
%% \usepackage{graphicx}

%% for colors
%% must be loaded before tikz in order for options to take effect :/
\usepackage[dvipsnames]{xcolor}

%% \usepackage{tikz}
%% \usepackage{tikz-cd}


\newcommand{\fa}[1]{\forall\,{#1}.\ }
\newcommand{\te}[1]{\exists\,{#1}.\ }

\newcommand{\N}{\mathbb{N}}
\newcommand{\ms}[1]{\ensuremath{\textsf{#1}}}
\newcommand{\setof}[2]{\{{#1}~|~{#2}\}}

\newcommand{\todo}[1]{{\color{red}{#1}}}

\newtheorem{theorem}{Theorem}
\newtheorem{corollary}{Corollary}
\newtheorem{nontheorem}{Non-theorem}
\newenvironment{refutation}
  {\vspace{0.5em}\noindent\textit{Refutation.}}
  {\hfill$\square$\vspace{0.5em}}


\begin{document}

\title{Consistency as Monotonicity}
\author{Michael Arntzenius}
\date{}
\maketitle


\section{Abstract}

We sketch a very general model of concurrent computation: a set of functions
iteratively applied in a nondeterministic order. By imposing some restrictions,
most importantly monotonicity, we ensure eventual consistency or determinism.


\section{Less abstract}

A program $P$ is a set of monotone maps on a poset $S$ of program states with a
least element (initial state) $\bot$. A state is ``fixed'' if it is a fixed
point of every function in $P$. A scheduling of a program is an infinite
sequence of elements of $P$. A scheduling is fair if it visits every $f \in P$
infinitely often. I show that a program either:

\begin{enumerate}
\item Diverges: no scheduling reaches a fixed state.
\item Converges: every fair scheduling reaches the same fixed state.
\end{enumerate}


\section{Definitions}

Fix $S$, a partially ordered set with a least element $\bot$.

We write $\vec{s} : P$ to indicate a finite sequence $\vec{s}$ such that $\fa{i} s_i \in P$.

\newcommand{\seq}[1]{{\vec{#1}_*}}

Given a sequence $\vec{f} = f_0, ..., f_n$, let $\seq{f} = f_n \circ f_{n-1}
\circ ... \circ f_1 \circ f_0$.

Consider sets $P,Q$ of monotone maps $S \to S$. Preorder such sets by:
\[
\begin{array}{ccccc}
P \le Q
&\text{iff}& \fa{f \in P} \te{\vec{s} : Q} f \le \seq{s}
&\text{or equivalently}& \fa{\vec{s} : P} \te{\vec{t} : Q}
{\seq{s} \le \seq{t}}\\
P \simeq Q &\text{iff}& P \le Q \wedge Q \le P
\end{array}
\]

\todo{Could I instead (or additionally) consider sets closed under composition
  and identity? Hm...}

\newcommand{\prefix}[1]{\ensuremath{\text{prefix}(#1)}}
\newcommand{\fixed}[1]{\ensuremath{\text{fixed}(#1)}}
\newcommand{\reach}[1]{\ensuremath{\text{reachable}(#1)}}

Let $\reach{P} = \setof{x \in S}{\te{\vec{s} : P} x = \seq{s}\; \bot}$. Call
these the reachable points of $P$.

Let $\prefix{P} = \setof{x \in S}{\fa{f \in P} f\; x \le x}$. Call these the prefix points of $P$.

Let $\fixed{P} = \setof{x \in S}{\fa{f \in P} f\;x = x}$. Call these the fixed
points of $P$. Note that all fixed points are prefix points.

\newcommand{\eval}{\Downarrow}
\newcommand{\diverge}{\not\eval}

Write $P \eval x$ (``$P$ evaluates to $x$'') for $x \in \reach{P} \cap
\prefix{P}$.

Write $P \diverge$ (``$P$ diverges'') for $\reach{P} \cap \prefix{P} = \emptyset$.


\section{Theorems}

Initially, I conjectured the following:

\begin{nontheorem}
  If $P \simeq Q$ then $\fixed{P} = \fixed{Q}$.
\end{nontheorem}

However, this is false:

\begin{refutation}
  Let $S = \N$ ordered $0 < 1 < 2 < ...$. Let $P = \{id\}$. Let $Q = \{id, (x
  \mapsto x - 1)\}$, where $0 - 1 = 0$. Then $P \simeq Q$; the only interesting
  case is $(x \mapsto x - 1) \le id$. Yet $\fixed{P} = \N$, but $\fixed{Q} =
  \{0\}$.
\end{refutation}

The idea of prefix points as a useful generalization of fixed points was
introduced to me by Paul Blain Levy. Prefix points turn out to solve this
problem:

\begin{theorem}
  If $P \le Q$ then $\prefix{Q} \subseteq \prefix{P}$.
  \label{thm:prefix-antitone}
\end{theorem}

\begin{proof}
  Let $x \in \prefix{Q}$ and $f \in P$. We wish to show $f\;x \le x$. By $P \le
  Q$ we have $f \le \seq{s}$ for some $\vec{s} : Q$. So $f\;x \le \seq{s}\;x$.
  Since $x \in \prefix{Q}$, $\seq{s}\;x \le x$. So $f\; x \le x$.
\end{proof}

\todo{Maybe the ordering on programs is the opposite of what it should be?}

\begin{corollary}
  If $P \simeq Q$, then $\prefix{P} = \prefix{Q}$.
  \label{cor:simeq->prefix}
\end{corollary}

This is interesting, but isn't it fixed points we really care about? As it turns
out, there is a powerful connection between prefix points and fixed points:

\begin{theorem}
  A least prefix point is a least fixed point.
\end{theorem}

\begin{proof}
  Since all fixed points are prefix points, a least prefix point is less than
  all fixed points. So it suffices to show that a least prefix point is a fixed
  point.

  Let $x$ be a least prefix point. Consider any $f \in P$. We wish to show $f\;x
  = x$. We have $f\; x \le x$. We will show that $f\;x$ is also a prefix point.
  From this, since $x$ is the least prefix point, we have $x \le f\;x$ and
  therefore $f\;x = x$.

  Consider any $g \in P$. We wish to show that $g\;(f\;x) \le f\; x$. We have
  $f\;x \le x$. Monotonically, $g\;(f\;x) \le g\;x$. Since $x$ is a prefix
  point, $g\;x \le x$. Transitively, $g\;(f\;x) \le x$ as desired.
\end{proof}

Now we consider the interaction of reachability and prefix-ness:

\begin{theorem}
  If $x \in \reach{P}$ and $y \in \prefix{P}$ then $x \le y$.
  \label{thm:reachable<=prefix}
\end{theorem}

\begin{proof}
  Since $x$ is reachable, $x = \seq{s}\; \bot$. Induct on $\vec{s}$. Base case:
  $\bot \le y$. Inductive hypothesis: $\seq{t}\; \bot \le y$. To show
  inductively: $f\; (\seq{t}\; \bot) \le y$ for $f \in P$. By our IH and
  monotonicity of $f$, $f\;(\seq{t}\; \bot) \le f\; y$. By $y \in \prefix{P}$,
  $f\;y \le y$. Transitively $f\; (\seq{t}\; \bot) \le y$ as desired.
\end{proof}

\begin{corollary}
  If $P \eval x$ then $x$ is the least fixed point of $P$.
\end{corollary}

\begin{corollary}[Consistency]
  If $P \eval x$ and $P \eval y$ then $x = y$.
\end{corollary}

\todo{Couldn't I prove a more traditional ``confluence'' theorem as well?}

\begin{theorem}[Equivalence]
  If $P \simeq Q$ and $P \eval x$ then $Q \eval x$.
\end{theorem}

\begin{proof}
  Since $x \in \prefix{P}$ and $Q \le P$, by Theorem \ref{thm:prefix-antitone}
  we know $x \in \prefix{Q}$. So it suffices to show that $x \in \reach{Q}$.

  $x = \seq{s}\;\bot$ for some $\vec{s} : P$. So by $P \le Q$ there is some
  $\vec{t} : Q$ such that $\seq{s} \le \seq{t}$ and so $x = \seq{s}\; \bot \le
  \seq{t}\;\bot$. Since $\seq{t}\;\bot \in \reach{Q}$ and $x \in \prefix{Q}$, by
  Theorem \ref{thm:reachable<=prefix} we have $\seq{t}\;\bot \le x$. By
  antisymmetry $x = \seq{t}\;\bot$ and is thus reachable in $Q$.
\end{proof}

\end{document}
